\documentclass[a4paper,11pt]{article}
\usepackage[utf8]{inputenc}
\usepackage[russian]{babel}
\usepackage{hyperref}

\author{Олег Смирнов\\
\texttt{oleg.smirnov@gmail.com}}
\date{\today}
\title{Алгоритмы: построение и анализ -- программа курса}

\begin{document}
\maketitle
\begin{abstract}
Программа базируется на курсе ``6.046: Introduction to Algorithms'' Массачусетского
технологического института\footnote{\href{http://goo.gl/jIOiq}
{http://ocw.mit.edu/ ... /6-046j-introduction-to-algorithms-sma-5503-fall-2005}}, 
на одноимённом учебнике Т. Кормена, Ч Лейзерсона, Р. Ривеста и К. Штайна, а также
учебниках ``Фундаментальные алгоритмы на C++'' Р. Седжвика и ``Программирование: теоремы
и задачи'' А. Шеня.
\end{abstract}
\section*{Раздел 1. Анализ алгоритмов}
\begin{itemize}
\item Тема 1.1. Анализ алгоритмов. Сортировка вставкой (Insertion sort) и сортировка слиянием (Merge sort). Бинарный поиск
\item Тема 1.2. Асимптотическая нотация. Анализ рекуррентные соотношений. Основная теорема
\item Тема 1.3. Парадигма ``Разделяй и властвуй''. Быстрое возведение в степень. Числа Фибоначчи. Алгоритм Евклида
\item Тема 1.4. Стеки и очереди. Дерево отрезков. Корневая эвристика
\item Тема 1.5. Пирамидальная сортировка (Heapsort). Очереди с приоритетами
\item Тема 1.6. Сортировка Quicksort. Рандомизированные алгоритмы
\item Тема 1.7. Сортировка за линейное время. Сортировка подсчётом (Counting sort). Поразрядная сортировка (Radix sort)
\end{itemize}
\section*{Раздел 2. Разработка алгоритмов}
\begin{itemize}
\item Тема 2.1. Хэширование. Хэш-функции. Идеальное хэширование
\item Тема 2.2. Двоичные поисковые деревья. Обход дерева и варианты записи дерева. Связь с Quicksort
\item Тема 2.3. Сбалансированные деревья. Пример реализации вставки в AVL
\item Тема 2.4. Декартово дерево (Treap)
\item Тема 2.5. Амортизационный анализ. Реализация динамических таблиц (Vector) и cистемы непересекающихся множеств (Disjoint-set)
\item Тема 2.6. Задача минимума на отрезке (RMQ) и решение корневой декомпозицией. Offline и online версии.
\item Тема 2.7. Задача наименьшего общего предка (LCA). Сведение LCA к RMQ
\item Тема 2.8. Динамическое программирование I. Задача поиска наибольшей общей подстроки (LCS)
\item Тема 2.9. Динамическое программирование II. Задача о ранце и о покрытии шахматной доски
\end{itemize}
\section*{Раздел 3. Алгоритмы на графах}
\begin{itemize}
\item Тема 3.1. Графы. Варианты записи графа. Поиск циклов. Поиск в ширину (BFS), в глубину (DFS) и топологическая сортировка. Двудольный граф.
\item Тема 3.2. Жадные алгоритмы. Задача минимального остова (MST). Алгоритм Прайма. Алгоритм Краскала
\item Тема 3.3. Кратчайшие пути I. Свойства и применение. Алгоритм Дейкстры
\item Тема 3.4. Кратчайшие пути II. Алгоритм Беллмана-Форда. Задача разностных ограничений
\item Тема 3.5. Кратчайшие пути III. Все пары вершин. Алгоритм Флойда-Уоршолла
\item Тема 3.6. Алгоритм Джонсона. Связность графа
\item Тема 3.7. Порядковые статистики. Динамические порядковые статистики
\end{itemize}
\section*{Раздел 4. Дополнительные темы}
\begin{itemize}
\item Тема 4.1. Вычислительная геометрия I. Векторное представление. Выпуклая оболочка. Метод заметания
\item Тема 4.2. Вычислительная геометрия II. Точка в полигоне. Площадь полигона
\item Тема 4.3. Строковые алгоритмы I. Префикс-функция. Z-функция. Хэширование
\item Тема 4.4. Строковые алгоритмы II. Бор. Алгоритм Рабина-Карпа
\item Тема 4.5. Максимальные поток и минимальный разрез. Алгоритм Форда-Фалкенсона. Парасочетания
\end{itemize}
\end{document}
